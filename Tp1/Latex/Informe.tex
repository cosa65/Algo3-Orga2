\documentclass[spanish,a4paper]{article}

% Paquetes generales
\usepackage{ifthen}
\usepackage{amssymb}
\usepackage{multicol}
\usepackage[absolute]{textpos}

% Cuestiones de enunciados

% Primero definiciones de cosas al estilo title, author, date
\def\materia#1{\gdef\@materia{#1}}
\def\@materia{M\'{e}todos Num\'{e}ricos}
\def\lamateria{\@materia}

\def\cuatrimestre#1{\gdef\@cuatrimestre{#1}}
\def\@cuatrimestre{No especifi\'o el cuatrimestre}
\def\elcuatrimestre{\@cuatrimestre}

\def\anio#1{\gdef\@anio{#1}}
\def\@anio{No especifi\'o el anio}
\def\elanio{\@anio}

\def\fecha#1{\gdef\@fecha{#1}}
\def\@fecha{\today}
\def\lafecha{\@fecha}

\def\nombre#1{\gdef\@nombre{#1}}
\def\@nombre{No especific'o el nombre}
\def\elnombre{\@nombre}

\def\practica#1{\gdef\@practica{#1}}
\def\@practica{No especifi\'o el n\'umero de pr\'actica}
\def\lapractica{\@practica}

% Esta macro convierte el numero de cuatrimestre a palabras
\newcommand{\cuatrimestreLindo}{
  \ifthenelse{\equal{\elcuatrimestre}{1}}
  {Primer cuatrimestre}
  {\ifthenelse{\equal{\elcuatrimestre}{2}}
  {Segundo cuatrimestre}
  {Verano}}
}

\newcommand{\depto}{{UBA -- Facultad de Ciencias Exactas y Naturales --
      Departamento de Computaci\'on}}

\newcommand{\titulopractica}{
  \centerline{\depto}
  \vspace{1ex}
  \centerline{{\Large\lamateria}}
  \vspace{0.5ex}
  \centerline{\cuatrimestreLindo de \elanio}
  \vspace{2ex}
  \centerline{{\huge Pr\'actica \lapractica -- \elnombre}}
  \vspace{5ex}
  \arreglarincisos
  \newcounter{ejercicio}
  \newenvironment{ejercicio}{\stepcounter{ejercicio}\textbf{Ejercicio
      \theejercicio}%
    \renewcommand\@currentlabel{\theejercicio}%
  }{\vspace{0.2cm}}
}  

\newcommand{\titulotp}{
  \centerline{\depto}
  \vspace{1ex}
  \centerline{{\Large\lamateria}}
  \vspace{0.5ex}
  \centerline{\cuatrimestreLindo de \elanio}
  \vspace{0.5ex}
  \centerline{\lafecha}
  \vspace{2ex}
  \centerline{{\huge\elnombre}}
  \vspace{5ex}
}

\usepackage[spanish]{babel}
\selectlanguage{spanish}
\usepackage[utf8]{inputenc}
%\usepackage{bbm}
\usepackage{framed}


\newcommand{\comen}[2]{%
\begin{framed}
\noindent \textsf{#1:} #2
\end{framed}
}

% **************************************************************************
%
%  Package 'caratula', version 0.5 (para componer caratulas de TPs del DC).
%
%  En caso de dudas, problemas o sugerencias sobre este package escribir a
%  Brian J. Cardiff (bcardif arroba gmail.com).
%  Nico Rosner (nrosner arroba dc.uba.ar).
%
% **************************************************************************

% ----- Informacion sobre el package para el sistema -----------------------

\NeedsTeXFormat{LaTeX2e}
\ProvidesPackage{caratula}[2013/08/04 v0.5 Para componer caratulas de TPs del DC]
\RequirePackage{ifthen}
\usepackage[pdftex]{graphicx}

% ----- Imprimir un mensajito al procesar un .tex que use este package -----

\typeout{Cargando package 'caratula' v0.5 (2013/08/04)}

% ----- Algunas variables --------------------------------------------------

\let\Materia\relax
\let\Submateria\relax
\let\Titulo\relax
\let\Subtitulo\relax
\let\Grupo\relax
\let\Fecha\relax
\let\Logoimagefile\relax
\newcommand{\LabelIntegrantes}{}
\newboolean{showLU}
\newboolean{showEntregas}
\newboolean{showDirectores}

% ----- Comandos para que el usuario defina las variables ------------------

\def\materia#1{\def\Materia{#1}}
\def\submateria#1{\def\Submateria{#1}}
\def\titulo#1{\def\Titulo{#1}}
\def\subtitulo#1{\def\Subtitulo{#1}}
\def\grupo#1{\def\Grupo{#1}}
\def\fecha#1{\def\Fecha{#1}}
\def\logoimagefile#1{\def\Logoimagefile{#1}}

% ----- Token list para los integrantes ------------------------------------

\newtoks\intlist\intlist={}

\newtoks\intlistSinLU\intlistSinLU={}

\newcounter{integrantesCount}
\setcounter{integrantesCount}{0}
\newtoks\intTabNombre\intTabNombre={}
\newtoks\intTabLU\intTabLU={}
\newtoks\intTabEmail\intTabEmail={}

\newcounter{directoresCount}
\setcounter{directoresCount}{0}
\newtoks\direcTabNombre\direcTabNombre={}
\newtoks\direcTabEmail\direcTabEmail={}

% ----- Comando para que el usuario agregue integrantes --------------------

\def\integrante#1#2#3{%
    \intlist=\expandafter{\the\intlist\rule{0pt}{1.2em}#1&#2&\tt #3\\[0.2em]}%
    \intlistSinLU=\expandafter{\the\intlistSinLU\rule{0pt}{1.2em}#1 & \tt #3\\[0.2em]}%
    %
    \ifthenelse{\value{integrantesCount} > 0}{%
        \intTabNombre=\expandafter{\the\intTabNombre & #1}%
        \intTabLU=\expandafter{\the\intTabLU & #2}%
        \intTabEmail=\expandafter{\the\intTabEmail & \tt #3}%
    }{
        \intTabNombre=\expandafter{\the\intTabNombre #1}%
        \intTabLU=\expandafter{\the\intTabLU #2}%
        \intTabEmail=\expandafter{\the\intTabEmail \tt #3}%
    }%
    \addtocounter{integrantesCount}{1}%
}

\def\director#1#2{%
    \ifthenelse{\value{directoresCount} > 0}{%
        \direcTabNombre=\expandafter{\the\direcTabNombre & #1}%
        \direcTabEmail=\expandafter{\the\direcTabEmail & \tt #2}%
    }{
        \direcTabNombre=\expandafter{\the\direcTabNombre #1}%
        \direcTabEmail=\expandafter{\the\direcTabEmail \tt #2}%
    }%
    \addtocounter{directoresCount}{1}%
}

% ----- Macro para generar la tabla de integrantes -------------------------

\newcommand{\tablaIntegrantes}{\ }

\newcommand{\tablaIntegrantesVertical}{%
\ifthenelse{\boolean{showLU}}{%
    \begin{tabular}[t]{| l @{\hspace{4ex}} c @{\hspace{4ex}} l|}
        \hline
        \multicolumn{1}{|c}{\rule{0pt}{1.2em} \LabelIntegrantes} & LU &  \multicolumn{1}{c|}{Correo electr\'onico} \\[0.2em]
        \hline \hline
        \the\intlist
        \hline
    \end{tabular}
}{
    \begin{tabular}[t]{| l @{\hspace{4ex}} @{\hspace{4ex}} l|}
        \hline
        \multicolumn{1}{|c}{\rule{0pt}{1.2em} \LabelIntegrantes} &  \multicolumn{1}{c|}{Correo electr\'onico} \\[0.2em]
        \hline \hline
        \the\intlistSinLU
        \hline
    \end{tabular}
    }%
}

\newcommand{\tablaIntegrantesHorizontal}{%
    \begin{tabular}[t]{ *{\value{integrantesCount}}{c} }
    \the\intTabNombre \\%
\ifthenelse{\boolean{showLU}}{
    \the\intTabLU \\%
}{}
    \the\intTabEmail %
    \end{tabular}%
}

\newcommand{\tablaDirectores}{%
\ifthenelse{\boolean{showDirectores}}{%
    \bigskip
    Directores

    \smallskip
    \begin{tabular}[t]{ *{\value{directoresCount}}{c} }
    \the\direcTabNombre \\%
    \the\direcTabEmail %
    \end{tabular}%
}{}%
}

\newcommand{\tablaEntregas}{%
\ifthenelse{\boolean{showEntregas}}{%
  \bigskip%
  \begin{tabular}[t]{|l p{3.5cm} p{1.5cm}|}%
  \hline%
  \rule{0pt}{1.2em} Instancia & Docente & Nota \\[0.2em] %
  \hline%
  \hline%
  \rule{0pt}{1.2em} Primera entrega & & \\[0.2em] %
  \hline%
  \rule{0pt}{1.2em} Segunda entrega & & \\[0.2em] %
  \hline%
  \end{tabular}%
}{}%
}

% ----- Codigo para manejo de errores --------------------------------------

\def\se{\let\ifsetuperror\iftrue}
\def\ifsetuperror{%
    \let\ifsetuperror\iffalse
    \ifx\Materia\relax\se\errhelp={Te olvidaste de proveer una \materia{}.}\fi
    \ifx\Titulo\relax\se\errhelp={Te olvidaste de proveer un \titulo{}.}\fi
    \edef\mlist{\the\intlist}\ifx\mlist\empty\se%
    \errhelp={Tenes que proveer al menos un \integrante{nombre}{lu}{email}.}\fi
    \expandafter\ifsetuperror}

\def\aftermaketitle{%
  \setcounter{page}{1}
}

% ----- \maketitletxt correspondiente a la versi�n v0.2.1 (texto v0.2 + fecha ) ---------

\def\maketitletxt{%
    \ifsetuperror\errmessage{Faltan datos de la caratula! Ingresar 'h' para mas informacion.}\fi
    \thispagestyle{empty}
    \begin{center}
    \vspace*{\stretch{2}}
    {\LARGE\textbf{\Materia}}\\[1em]
    \ifx\Submateria\relax\else{\Large \Submateria}\\[0.5em]\fi
    \ifx\Fecha\relax\else{\Large \Fecha}\\[0.5em]\fi
    \par\vspace{\stretch{1}}
    {\large Departamento de Computaci\'on}\\[0.5em]
    {\large Facultad de Ciencias Exactas y Naturales}\\[0.5em]
    {\large Universidad de Buenos Aires}
    \par\vspace{\stretch{3}}
    {\Large \textbf{\Titulo}}\\[0.8em]
    {\Large \Subtitulo}
    \par\vspace{\stretch{3}}
    \ifx\Grupo\relax\else\textbf{\Grupo}\par\bigskip\fi
    \tablaIntegrantes
    \end{center}
    \vspace*{\stretch{3}}
    \newpage\aftermaketitle}

% ----- \maketitletxtlogo correspondiente v0.2.1 (texto con fecha y logo) ---------

\def\maketitletxtlogo{%
    \ifsetuperror\errmessage{Faltan datos de la caratula! Ingresar 'h' para mas informacion.}\fi
    \thispagestyle{empty}
    \begin{center}
    \ifx\Logoimagefile\relax\else\includegraphics{\Logoimagefile}\fi \hfill \includegraphics{logo_dc.jpg}\\[1em]
    \vspace*{\stretch{2}}
    {\LARGE\textbf{\Materia}}\\[1em]
    \ifx\Submateria\relax\else{\Large \Submateria}\\[0.5em]\fi
    \ifx\Fecha\relax\else{\large \Fecha}\\[0.5em]\fi
    \par\vspace{\stretch{1}}
    {\large Departamento de Computaci\'on}\\[0.5em]
    {\large Facultad de Ciencias Exactas y Naturales}\\[0.5em]
    {\large Universidad de Buenos Aires}
    \par\vspace{\stretch{3}}
    {\Large \textbf{\Titulo}}\\[0.8em]
    {\Large \Subtitulo}
    \par\vspace{\stretch{3}}
    \ifx\Grupo\relax\else\textbf{\Grupo}\par\bigskip\fi
    \tablaIntegrantes
    \end{center}
    \vspace*{\stretch{4}}
    \newpage\aftermaketitle}

% ----- \maketitlegraf correspondiente a la versi�n v0.3 (gr�fica) -------------

\def\maketitlegraf{%
    \ifsetuperror\errmessage{Faltan datos de la caratula! Ingresar 'h' para mas informacion.}\fi
%
    \thispagestyle{empty}

    \ifx\Logoimagefile\relax\else\includegraphics{\Logoimagefile}\fi \hfill \includegraphics{logo_dc.jpg}

    \vspace*{.06 \textheight}

    \noindent \textbf{\huge \Titulo}  \medskip \\
    \ifx\Subtitulo\relax\else\noindent\textbf{\large \Subtitulo} \\ \fi%
    \noindent \rule{\textwidth}{1 pt}

    {\noindent\large\Fecha \hspace*\fill \Materia} \\
    \ifx\Submateria\relax\else{\noindent \hspace*\fill \Submateria}\fi%

    \medskip%
    \begin{center}
        \ifx\Grupo\relax\else\textbf{\Grupo}\par\bigskip\fi
        \tablaIntegrantes

        \tablaDirectores

        \tablaEntregas
    \end{center}%
    \vfill%
%
    \begin{minipage}[t]{\textwidth}
        \begin{minipage}[t]{.55 \textwidth}
            \includegraphics{logo_uba.jpg}
        \end{minipage}%%
        \begin{minipage}[b]{.45 \textwidth}
            \textbf{\textsf{Facultad de Ciencias Exactas y Naturales}} \\
            \textsf{Universidad de Buenos Aires} \\
            {\scriptsize %
            Ciudad Universitaria - (Pabell\'on I/Planta Baja) \\
                Intendente G\"uiraldes 2160 - C1428EGA \\
            Ciudad Aut\'onoma de Buenos Aires - Rep. Argentina \\
                Tel/Fax: (54 11) 4576-3359 \\
            http://www.fcen.uba.ar \\
            }
        \end{minipage}
    \end{minipage}%
%
    \newpage\aftermaketitle}

% ----- Reemplazamos el comando \maketitle de LaTeX con el nuestro ---------
\renewcommand{\maketitle}{\maketitlegraf}

% ----- Dependiendo de las opciones ---------
%
% opciones:
%   txt     : caratula solo texto.
%   txtlogo : caratula txt con logo del DC y del grupo (opcional).
%   graf    : (default) caratula grafica con logo del DC, UBA y del grupo (opcional).
%
%\@makeother\*% some package redefined it as a letter (as color.sty)
%
% Layout general de la caratula
%
\DeclareOption{txt}{\renewcommand{\maketitle}{\maketitletxt}}
\DeclareOption{txtlogo}{\renewcommand{\maketitle}{\maketitletxtlogo}}
\DeclareOption{graf}{\renewcommand{\maketitle}{\maketitlegraf}}
%
% Etiqueta Autores o Integrantes
%
\DeclareOption{integrante}{\renewcommand{\LabelIntegrantes}{Integrante}}
\DeclareOption{autor}{\renewcommand{\LabelIntegrantes}{Autor}}
%
% Formato tabla de integrantes
%
\DeclareOption{intVert}{\renewcommand{\tablaIntegrantes}{\tablaIntegrantesVertical}}
\DeclareOption{intHoriz}{\renewcommand{\tablaIntegrantes}{\tablaIntegrantesHorizontal}}
\DeclareOption{conLU}{\setboolean{showLU}{true}}
\DeclareOption{sinLU}{\setboolean{showLU}{false}}
\DeclareOption{conEntregas}{\setboolean{showEntregas}{true}}
\DeclareOption{sinEntregas}{\setboolean{showEntregas}{false}}
\DeclareOption{showDirectores}{\setboolean{showDirectores}{true}}
\DeclareOption{hideDirectores}{\setboolean{showDirectores}{false}}
%
% Opciones predeterminadas
%
\ExecuteOptions{intVert}%
\ExecuteOptions{graf}%
\ExecuteOptions{integrante}%
\ExecuteOptions{conLU}%
\ExecuteOptions{hideDirectores}%
\ExecuteOptions{sinEntregas}%
%
\ProcessOptions\relax

\begin{document}

\materia{}

\titulo{M\'{e}todos Num\'{e}ricos}
\subtitulo{TP 1: "No creo que a  \'{e}l le gustara eso"}

\fecha{\today}


\grupo{}

\integrante{Fosco, Martin Esteban}{449/13}{mfosco2005@yahoo.com.ar}
\integrante{Minces Müller, Javier Nicolás}{231/13}{javijavi1994@gmail.com}
\integrante{Chibana, Christian Tomokazu De La Vega Jr.}{}{tomistomus@LaMilagrosa.com.jp}



\maketitle

\clearpage

\nombre{\LARGE }

\section{Resumen}

En este trabajo implementamos una solución al problema planteado en el tp, encontrar una manera de determinar qué sanguijuelas pegadas al parabrisas son peligrosas y eliminarlas, usando una representación matricial del sistema de ecuaciones que me permite hallar la temperatura del parabrisas en cada punto (con una precisión determinada por la granularidad) tomando como incógnitas a, justamente, cada punto del parabrisas.\newline
Para resolver dicho sistema de ecuaciones recurrimos a técnicas matemáticas basadas en métodos de resolución de sistemas, y dividimos en distintos módulos (structs) con funciones específicas al parabrisas y a la matriz del sistema de ecuaciones para facilitar la comprensión (tanto de los docentes como de nosotros mismos) de la implementación hecha.%para facilitar la comprensión? en serio?

\newpage

\section{Introducci\'{o}n Te\'{o}rica}

El objetivo de este trabajo es resolver el sistema de ecuaciones 

%(esto dice que tiene que ser breve, probablemente mucho de esto vaya a desarrollo)
%(saquen algunos "de esta forma" que es un vicio que tengo, lo que saqué fueron los "bla bla bla")

 Como dato, se tenía que cada punto del parabrisas satisfacía la ecuación de Laplace. %<<Insert equation here>>http://upload.wikimedia.org/math/4/4/7/447c028a9431b8c38fd5a882911b430d.png

Para poder modelar el parabrisas se tomó una discretización, con granularidad variable. De esta forma, el parabrisas quedó representado con una matriz de a/h+1 filas y l/h+1 columnas, siendo a el ancho en metros del parabrisas original, l su largo y h la granularidad elegida y l su largo. A esta matriz la llamaremos "matriz parabrisas".

Aplicando el modelo de aproximación por diferencias finitas se puede ver que la temperatura de cada punto desconocido puede expresarse como el promedio de cada uno de los puntos adyacentes. Es decir, la temperatura de cada punto del parabrisas se puede expresar como una ecuación lineal. %<<insert ecuación lineal genérica con (i,j) here>>.

Considerando todos los puntos desconocidos de la matriz, se obtuvo entonces un sistema de ecuaciones lineales de a+l x a+l, siendo a y l el ancho y el largo de la matriz parabrisas, respectivamente.

%Matriz de ejemplo, suma 80 puntos, no sé si escrinshoteada

Para resolver nuestro sistema de ecuaciones (expresado en forma de una matriz) se usó el método conocido como "Eliminación Gaussiana", que nos permite triangular una matriz de forma relativamente eficiente (es decir llegar de una matriz común a una triangular superior) recurriendo a la resta entre filas de la misma matriz (con un multiplicador que sirva para reducir los valores por debajo de la diagonal a 0).


Una vez triangulada la matriz, pueden hallarse los valores del vector incógnita más fá reemplazando de abajo hacia arriba con los valores conocidos.

La segunda parte del problema consistía en encontrar un algoritmo que permitiera elegir qué sanguijuelas eliminar para que la temperatura del punto crítico del vidrio se mantuviera por debajo de 235º, eliminando la menor cantidad posible. El algoritmo %(meter "heurístico" en aluna parte) nuestro fue este:

%<<algoritmo en pseudocódigo here>>, aunque creo que necesito macros de algo2 para ponerlo

\section{Desarrollo}


Nuestro trabajo apunta a resolver un problema que afrontamos dividiendo en dos partes. La primera parte del problema consistía en hallar una forma de representar el calor causado por una sanguijuelas en un parabrisas rectangular en un momento determinado representando dicho parabrisas como una matriz y utilizando el lenguaje C++. Algunos puntos eran conocidos, ya que el parabrisas tenía bordes con una temperatura constante, y fuentes de calor (sanguijuelas) con radio y temperatura a determinar por los parámetros r y t.

%sé que acá va lo de intentos fallidos y esas cosas, y eso que dijo el tipo que pongamos en algún lado de por qué no guardamos solo el vector de 1s y -1s
Nuestro problema entonces se reducía a plantear una resolución
En un principio pensamos en utilizar una sola estructura para el problema, una matriz que generara la matriz a partir de los datos leídos en el archivo de entrada (que nos indica los datos relevantes para hallar la temperatura en cada punto del parabrisas). El problema que encontramos era que al crear la matriz, solo guardábamos sus celdas, perdiendo los datos originales del problema. De esta forma, no podíamos generar la matriz parabrisas sin el saber su ancho, para devolver los datos de la forma requerida. Guardar estos datos como parte de la estructura hubiera sido una solución forzada, que hubiera hecho que la matriz dejara de ser una estructura genérica para volverse específica de este problema. Por eso, optamos por una estructura parabrisas, que guardara todos los datos relevantes. %y si pongo una palabra más en este párrafo me mandan de nuevo alengua a aprender puntuación
%(esto no va acá ni en pedo)








Para almacenar la matriz se usaron dos métodos. El primero fue un vector de filas, donde cada fila era un vector con todos sus valores. Para el segundo aprovechamos la estructura banda de la matriz (¿explicar/definir?/desarrollar) para almacenar, de cada fila, solo los elementos que forman parte de la banda. Al triangular, la matriz mantiene su ancho de banda.










Para aprovechar la estructura banda se pensó en guardar únicamente un vector de valores booleanos %(si, esto es fruta)
que permitiera saber para cada celda de la matriz parabrisas si su valor era conocido o desconocido, ya que solo con ese dato era posible saber el valor de toda la fila de la otra matriz. Sin embargo, esta estructura no puede mantenerse al triangular, ya que el valor de la fila toma valores que no dependen únicamente de la celda que representa. %Por ahí deberíamos haber guardado así: 1/4, cantidad de 0 que siguen, valores que siguen (no digo que pongamos eso, es solo una reflexión)

Finalmente, para implementar la matriz banda, no creamos otra estructura, sino que modificamos la estructura matriz ya existente, de forma que incluyera un booleano que dijera si era una matriz banda, y un entero que indicara el ancho de banda. De esta forma, todas las operaciones de la matriz podían utilizarse para la matriz banda, con la única exepción de la función que permite definir un valor en una celda y la que busca el valor de una celda determinada, en las que hubo que separar en dos casos.

Al implementar esto, hubo que modificar el algoritmo que crea la matriz a partir de un parabrisas para que solo definiera los elementos de la banda.También hubo que modificar indirectamente la eliminación gaussiana, ya que al restar filas solo debían tenerse n cuenta en las celdas que formaran parte de la banda.  %les importa eso?

%También hay que poner acá como se interpreta el archivo generado por el algoritmo de eliminación de las sanguijuelas. Y que para para probarlo usamos solo la matriz banda

%Y en algún lado (acá o en conclusiones) lo de la tolerancia (importante, porque el tipo recomendó trabajar con tolerancia, pero poner =0 es lo mismo, así que podemos poner que lo dejamos en =0 por eso). También pivoteo. 

\section{Resultados}

Tiempo de ejecución según la granularidad
hacer grafico con esto:
h banda	sin banda\\
0.5  223.142 -	\\
1  13.774	462.923\\
2  0.893	13.245\\
2.5  0.38	4.374\\
5  0.037	0.153\\
10  0.009	0.062\\

%Aunque los tiempos son posta, no me acuerdo de qué test eran, aunque puedo buscarlo, siempre podemos frutear

%<<Copiar y pegar un LSD.out>>

%Y si les parece poco chamuyamos con lo de que cambiar la granularidad afecta la temperatura del punto medio

%Consigna:
%Considerar al menos dos instancias de prueba, generando discretizaciones variando la
%granularidad para cada una de ellas y comparando el valor de la temperatura en el punto
%crítico. Se sugiere presentar gráficos de temperatura para los mismos, ya sea utilizando
%las herramientas provistas por la cátedra o implementando sus propias herramientas de
%graficación.
% Analizar el tiempo de cómputo requerido en función de la granularidad de la discretización, buscando un compromiso entre la calidad de la solución obtenida y el tiempo de cómputo requerido. Comparar los resultados obtenidos para las variantes propuestas en 1 y 2.
% Estudiar el comportamiento del método propuesto para la estimación de la temperatura
%en el punto crítico y para la eliminación de sanguijuelas.

\section{Discusi\'{o}n}

Algo de factorización LU? no tengo idea de qué va acá

\section{Conclusiones}
La banda es más rápida (¿cuanto?) y ocupa menos memoria (¿cuanto?) %(no, en serio?)

Reducir n veces la granularidad implica aumentar el tiempo de ejecución n%^
6 veces en el caso de la matriz común y n%^
4 en el caso de la banda %(ponele, hay comparar con los resultados) <-Puede ser que esto vaya en discusión y en las conclusiones solo algo más general?

El tiempo de ejecución no depende de la cantidad de sanguijuelas, salvo porque puede implicar más puntos determinados.

Algo del algoritmo de eliminación. Por ejemplo, que al eliminar cada sanguijuela baja la temperatura, salvo que haya dos en exactamente el mismo lugar. Eso habla bien del algoritmo. Algo de cuánto tarda, por ejemplo, comparando con un algortimo que asegure ser óptimo.

Qué tal si vemos cuanto tardan las diferentes partes del programa. %No es necesario hacerlo literalmente, solo decir que la eliminación gaussiana se come todo.


%Fosquii, te faltó pasar de hoja en Pautas
\section{Apéndice A}

%<<Copiar enunciado, lo haría yo pero se van a ilusionar con que de golpe tenemos 10 páginas>>
%y si quieren apéndice B: algoritmos en pseudocódigo \forall

\section{Referencias}

%Roulinsson, Rouli, "Como hacer el TP de Métodos for dummies", Ed. edificio dorado, 2014

%Eh, no sé, yo en el secundario siempre puse apuntes de clase y si quieren pongan el libro que sea el de la materia, aunque si lo toco para el final va a ser como :O

\end{document}